% ----------------------------------------------------------
\chapter{Results and Discussion}
% ----------------------------------------------------------

On the data collection part of this research, we evaluate the impact of a data aquisition model on AVs regarding local processing, storage and RAM usage (these parameters can also be used to estimate energy consumption). The experiment used the CARLA simulator in a client-server architecture and applied real-time data collection directly from the simulation server, as well as using application performance management tools.

Resulting data show that an extra data collection module at the vehicles offer a negligible 1.1\% increase in total memory usage  with direct sensor collection and a 2.6\% increase with application performance management (APM) data collection on the reference hardware.

The traffic monitoring experiment was designed to test the research question about the feasibility of traffic monitoring by the reuse of AV data for real-time traffic monitoring. Three "monitor vehicles" were placed in the simulation to collect GNSS and speedometer data. Two traffic conditions were simulated: light and heavy traffic. We show that it is feasible but we don't discuss network, identification and security concerns, important parameters in this topic, as that is reserved for future experiments.

The results obtained in the simulation by the proof-of-concept indicate that the proposed architecture could be applied in real world AVs; this form of data reuse can significantly improve ITS performance in its biggest challenges with minimal impact on current technology stack. Moreover, the reported experience with proof-of-concept allowed the identification of other promising research directions.

Our data transmission simulation expanded these concepts by integrating the CARLA simulator with communication network modeling through the B5GCyberTestV2X framework. This integration enabled realistic 5G network simulations for V2X communications, revealing that the bit rate required for basic cooperative perception (approximately 61.68 KB/s per vehicle) consumes only 0.049\% of a 5G antenna's capacity. This finding suggests that a single antenna could theoretically handle data from over 2000 vehicles simultaneously, confirming the technical feasibility of large-scale V2X implementations.

The development of the cooperative perception simulation framework for V2X security experiments represents a significant advancement in our research. By enabling vehicles to share sensor data through simulated 5G connections, we created a environment for testing cooperative perception under various security scenarios. The framework's modular architecture with separate components for data collection, fusion, and vehicle control facilitated detailed analysis of how vehicles can effectively share and utilize perception data.

Our cyber security experiments demonstrated the vulnerability of V2X systems to spoofing attacks. Through three detailed attack scenarios (Do Not Pass Warning, Vulnerable Road User Alerts, and Left Turn Assist), we simulated how malicious actors could compromise vehicle safety by injecting false information. The dataset created from these simulations provides valuable ground truth for developing and validating security countermeasures.

The vision model fine-tuning component of our research addressed the critical need for robust drone detection as part of a broader attack mitigation strategy. By creating a specialized dataset of 4,824 annotated images and finetuning a YOLOv8 model, we achieved high detection accuracy (mAP50 of approximately 0.9), demonstrating that visual confirmation can complement signal-based approaches in identifying potential threats.

Right now the amount of sensor data that would be transmitted, reflecting what we measured in usage of storage, is considerable. To tackle this problem, one of the possible research directions is the local pre-processing of the data in the vehicles prior to transmission and analyzing the impacts in relation to computing, like explored in \cite{THANDAVARAYAN2023103655}. Another involves proposing to improve the reliability of B5G in the presence of intermittent connectivity that can degrade data accuracy.

The B5GCyberTestV2X framework provides a starting platform for future research in V2X security, enabling more sophisticated simulations and the development of advanced defensive strategies. Through continued refinement of these tools and methodologies, the security and reliability of cooperative perception systems in autonomous driving could be significantly enhanced, making them more resilient against sophisticated cyber attacks and ensuring their robust operation in diverse real-world scenarios.

% The experiments conducted provide a foundation for understanding and furthering the capabilities of ITS technologies through AV data collection and traffic monitoring systems. The results encourage continued progression in cooperative perception and fusion technologies to address existing limitations within these systems. Further research may lead to the discovery of new avenues for smarter, more efficient transportation solutions.