% ----------------------------------------------------------
\chapter{Introduction}
\label{cap_intr}
% ----------------------------------------------------------

The recent advancements in autonomous vehicles and Intelligent Transportation System technologies offer a great potential to improve the landscape of logistics and transportation, resulting in a safer and more efficient traffic environment. This change is propelled by a convergence of technological innovations and can generate great progress in the field.

ITS have significantly improved transportation quality by using applications capable of monitoring, managing, and improving the transportation system. However, the large number of devices required to provide data to ITS applications has become a challenge in recent years, particularly the high installation and maintenance costs made broad deployment impracticable. Despite several advances in smart city research and the internet of things, research on ITS is still in the early stages. For example, data integration and reuse from different sources are not yet supported by the underlying infrastructure that provides data to ITS applications. 

Assuming a future where a share of vehicles are equipped with advanced sensors, computing systems, and artificial intelligence algorithms, these vehicles carry a significant number of sensors and have the potential to create a shared data infrastructure that can help improve traffic monitoring, performance, and safety.

Given this scenario, the following research questions will be evaluated: Is it possible to collect the Connected Vehicles data with minimal impact on the vehicle's embedded system? What is the feasibility to emulate a generic "plug and play" module that would work in any vehicle of this type? At scale, is it possible to use this data for improving vehicle sensing capabilities and traffic monitoring ?

In this sense, in order to experiment with data collection and AV shared perception techniques, this work contributes with: a proof of concept simulation environment for shared perception based on V2X Technologies, and a set of experiments using the developed POC for AV data collection and exchange. 


% ---
\section{Motivation}
% ---

In modern road traffic systems, inefficiencies often cause congestion and traffic jams, particularly during peak periods. The concentration of vehicles in specific locations and times intensify these problems, resulting in longer travel times and frustration for commuters. Additionally, delayed detection of local problems, such as broken traffic lights, vehicle collisions, or pavement damage, further exacerbates traffic disruptions. These incidents can escalate, causing significant inconvenience and safety hazards for road users if not addressed promptly.


On the other hand, autonomous vehicles offer a unique opportunity to utilize the vast amounts of data generated during their operation. Although much of this data is usually discarded after immediate use by the Simultaneous Localization and Mapping (SLAM) algorithm and control modules, it has substantial potential for reuse.

One possible approach is to analyze traffic behavior in real-time, which can provide valuable insights into traffic patterns, congestion dynamics, and driver behaviors. By utilizing this data, transportation authorities can improve route planning algorithms, resulting in more efficient traffic management and smoother navigation for commuters.

Furthermore, the wealth of data generated by autonomous vehicles presents a valuable resource for training purposes. This data can be used to create large datasets for machine learning algorithms, which can aid in the development and validation of advanced autonomous driving systems.  
% Additionally, the operation of autonomous vehicles allows for experimental continuous learning models, where algorithms can be iteratively refined and optimized based on real-world data, leading to ongoing improvements in performance and safety. 

Finally, the availability of open data from autonomous vehicles has big potential for urban reuse. This data can provide insights into urban mobility patterns, infrastructure usage, and environmental impacts. City planners, researchers, and developers can benefit from this data.

% ---
\section{Scientific contribution}
% ---

In addition to this work, the author has contributed to the fields of computer science and intelligent transportation systems, resulting in the following publications:

\begin{citacao}[english]
1. L. Weigang, L. Martins, N. Ferreira, C. Miranda, L. Althoff, W. Pessoa, M. Farias, R. Jacobi, M. Rincon, "Heuristic Once Learning for Image \& Text Duality Information Processing," 2022 IEEE Smartworld, Ubiquitous Intelligence \& Computing, Scalable Computing \& Communications, Digital Twin, Privacy Computing, Metaverse, Autonomous \& Trusted Vehicles, Haikou, China, 2022, pp. 1353-1359, doi:  10.1109/ SmartWorld-UIC-ATC-ScalCom-DigitalTwin-PriComp-Metaverse 56740.2022.00195,
\end{citacao}

\begin{citacao}[english]
2. C. Miranda, A. Silva, J. da Costa G. A. Santos, E. P. de Freitas, A. Vinel, "A virtual infrastructure model based on data reuse to support intelligent transportation system applications" in IEEE Access, vol. 13, pp. 40607-40620, 2025, doi: 10.1109/ACCESS.2025.3547160.
\end{citacao}

\begin{citacao}[english]
3. D. Alves da Silva, A. Silva, D. Limas, J. da Costa, L de Melo, C. Miranda,
G. Santos, A. Vinel, P. Mendes, S. Verhoeven, J. Voigt-Antons, E. de Freitas, "Spoofer detection framework for V2X systems via tensor-based DoA estimation and Yolo-based object detection", forthcoming in IEEE Access.
\end{citacao}

% ---
\section{Outline}
% ---

The rest of this work is structured in the following way: Chapter 2 presents the Theoretical Background for the experiments technology stack, Chapter 3 presents the Literature Review and related works. Chapter 4 will discuss about the proper research questions and the research methodology, presenting the proposed techniques and performance metrics to evaluate experiment behavior. Chapter 5 shows the results and impacts on the network behavior based on each selected architecture. Chapter 6 concludes this work by summarizing the research questions responses and proposing future works.

